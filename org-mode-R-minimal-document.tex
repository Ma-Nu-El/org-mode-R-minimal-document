% Created 2021-09-10 Fri 20:49
% Intended LaTeX compiler: pdflatex
\documentclass[11pt]{article}
\usepackage[utf8]{inputenc}
\usepackage[T1]{fontenc}
\usepackage{graphicx}
\usepackage{grffile}
\usepackage{longtable}
\usepackage{wrapfig}
\usepackage{rotating}
\usepackage[normalem]{ulem}
\usepackage{amsmath}
\usepackage{textcomp}
\usepackage{amssymb}
\usepackage{capt-of}
\usepackage{hyperref}
\usepackage{tikz}
\author{manuel.fuica.morales@gmail.com}
\date{10 September 2021}
\title{Minimal document example for plot generation using Org Mode with R and tikzDevice exported to pdf format.\\\medskip
\large suited for a Doom Emacs distribution.}
\hypersetup{
 pdfauthor={manuel.fuica.morales@gmail.com},
 pdftitle={Minimal document example for plot generation using Org Mode with R and tikzDevice exported to pdf format.},
 pdfkeywords={},
 pdfsubject={},
 pdfcreator={Emacs 27.1 (Org mode 9.5)}, 
 pdflang={English}}
\begin{document}

\maketitle
\setcounter{tocdepth}{2}
\tableofcontents


\section{About}
\label{sec:org86fbc1e}
\subsection{this tutorial}
\label{sec:org46a11b8}

\begin{quote}
I'd recommend to read the pdf version of this alongside the
orgmode code so you can have both the source and the final product.
\end{quote}

I write tutorials with the objective of reminding me later how to do the
stuff when I forgot, so they are written from a noob-ish POV. They are
more like personal notes with a touch of in-case-someone-else-reads-it. It's
on the web and it's not private anyway so\ldots{}

The objective of this file is to be a minimal example of a 'live'
document using orgmode capabilities to work with \texttt{R}.

I have used as guide mainly the following tutorials:
\begin{itemize}
\item \url{https://orgmode.org/worg/org-contrib/babel/languages/ob-doc-LaTeX.html}
\item \url{https://orgmode.org/worg/org-contrib/babel/languages/ob-doc-R.html}
\item \url{https://orgmode.org/worg/org-tutorials/org-R/org-R.html}
\item \url{https://github.com/erikriverson/org-mode-R-tutorial/blob/master/org-mode-R-tutorial.org}
\end{itemize}

\subsection{\texttt{tikzDevice} and \texttt{knitr}}
\label{sec:org3885aea}
\begin{itemize}
\item Use:
\begin{enumerate}
\item \texttt{file.R} -> \texttt{knitr} -> standalone \texttt{file.tex} file.
\item \texttt{file.R} -> \texttt{tikzDevice} -> \texttt{file.tex} file that only contains the
plots of the \texttt{file.R} program. The plots in the \texttt{file.tex} file are
written in \texttt{tikz}, whereas in the \texttt{file.R} the plots are written
in, redundantly, \texttt{R}.
\end{enumerate}
\end{itemize}


\begin{itemize}
\item both \texttt{tikzDevice} and \texttt{knitr} are \texttt{R} libraries that enable \texttt{tex} friendly
output of an \texttt{file.R}.
\item \texttt{tikzDevice} enables \texttt{R} to create a \texttt{file.tex} out of the plots of
a \texttt{file.R} file.
\item \texttt{knitr} instead, enables the whole \texttt{file.R} file to be exported into a
\texttt{file.tex} file, including the plots and the rest of the document.
You can use it to build \texttt{reports}, \texttt{beamer} presentations, etc.
\end{itemize}


\begin{enumerate}
\item \texttt{R code} -> \texttt{tikzDevice} -> file that can be processed by \texttt{LaTeX}, but it
\emph{only} contains \texttt{tikz} code. Requires a separate preamble in order for
\texttt{LaTeX} to sucessfully generate a \texttt{pdf} file from it.
\item \texttt{R code} -> \texttt{knitr} -> file that can be processed by \texttt{LaTeX}, and can be
a standalone file which generates a \texttt{report} or \texttt{beamer} presentation, etc.
\emph{Can} be directly proccesed by \texttt{LaTeX} and will sucessfully generate a
\texttt{pdf} file.
\end{enumerate}

\section{Software used}
\label{sec:orgd36d694}

\subsection{emacs}
\label{sec:org5ab3ceb}
\begin{verbatim}
emacs --version
\end{verbatim}

\begin{verbatim}
GNU Emacs 27.1
Copyright (C) 2020 Free Software Foundation, Inc.
GNU Emacs comes with ABSOLUTELY NO WARRANTY.
You may redistribute copies of GNU Emacs
under the terms of the GNU General Public License.
For more information about these matters, see the file named COPYING.
\end{verbatim}

\subsection{doom}
\label{sec:org5f20c50}
\begin{verbatim}
~/.emacs.d/bin/doom version
\end{verbatim}

\begin{verbatim}
> Executing 'doom version' with Emacs 27.1 at 2021-09-10 20:45:56
  Doom v3.0.0-alpha (HEAD -> develop 13958c81 2021-07-10 10:30:49 -0400)
\end{verbatim}

\subsection{org}
\label{sec:orgf2004f9}
\begin{verbatim}
org-version
\end{verbatim}

\begin{verbatim}
9.5
\end{verbatim}


\subsection{R}
\label{sec:org8b14564}
\begin{verbatim}
version
\end{verbatim}

\begin{verbatim}
               _                           
platform       x86_64-pc-linux-gnu         
arch           x86_64                      
os             linux-gnu                   
system         x86_64, linux-gnu           
status                                     
major          4                           
minor          1.1                         
year           2021                        
month          08                          
day            10                          
svn rev        80725                       
language       R                           
version.string R version 4.1.1 (2021-08-10)
nickname       Kick Things
\end{verbatim}


\section{Emacs configuration}
\label{sec:org10d3071}

You have to configure your \texttt{\textasciitilde{}/.emacs/} in order for it to work with \texttt{R}.

Doom Emacs does a pretty good job. You can read the docs at:
\begin{itemize}
\item \url{https://github.com/hlissner/doom-emacs/blob/develop/modules/lang/ess/README.org}
\end{itemize}


In short, at time of writing all you have to do is
\begin{verbatim}
(ess +lsp)               ; emacs speaks statistics
\end{verbatim}

in your \texttt{\textasciitilde{}/.doom.d/init.el}. And probably install some stuff via
\texttt{sudo apt install ...} but that's out of the scope of this document.
And remember to \texttt{\textasciitilde{}/.emacs.d/bin/doom sync} after you edit that file.

Once your Emacs is configured to work with \texttt{R} \ldots{}

\section{Simple plot}
\label{sec:orgd104ea8}
\subsection{dependencies}
\label{sec:orgd0b6790}
Your \texttt{file.tex} generated by orgmode should contain the \texttt{tikz} library.

This is represented by \texttt{\texttt{\textbackslash{}usepackage\{tikz\}}}
somewhere in the preamble --- before the
\texttt{\texttt{\textbackslash{}begin\{document\}}}
line of the exported \texttt{file.tex} ---. You accomplish this by putting
\begin{verbatim}
#+LATEX_HEADER: \usepackage{tikz}
\end{verbatim}
somewhere early in the document.

Also, at least in my case, my \texttt{R} distribution, installed via command line
using the typical \texttt{sudo apt install r-cran-...} did not come with the
\texttt{tikzDevice} library pre-installed --- I'm a complete noob with R so I
do not know if that's default ---. Credits to this thread for the help:
\begin{itemize}
\item \url{https://lightonphiri.org/blog/r-graphical-representation-installing-tikzdevice-package}
\end{itemize}

\subsection{codeblocks}
\label{sec:org488d9e7}

From within a shell --- ahem, \texttt{tmux}, ahem --- you have to
\begin{verbatim}
$ R
> install.packages('tikzDevice')
\end{verbatim}
then follow the prompts.

\begin{quote}
I mean, you can do that from within an orgmode code source block, but that
led to some issues that are not worth solving for the document's purpose.
\end{quote}

Once the \texttt{tikzDevice} library is installed, you can proceed to evaluate
the following code blocks --- using \texttt{C-c C-c} --- :

\begin{itemize}
\item Actually load the library
\end{itemize}
\begin{verbatim}
  library("tikzDevice")
\end{verbatim}

\begin{itemize}
\item And the magic part
\end{itemize}

\begin{verbatim}
  tikz(console=TRUE, width=4, height=4)
  plot(rnorm(100), rnorm(100))
  dummy <- dev.off()
\end{verbatim}

% Created by tikzDevice version 0.12.3.1 on 2021-09-10 20:49:03
% !TEX encoding = UTF-8 Unicode
\relax
\begin{tikzpicture}[x=1pt,y=1pt]
\definecolor{fillColor}{RGB}{255,255,255}
\path[use as bounding box,fill=fillColor,fill opacity=0.00] (0,0) rectangle (289.08,289.08);
\begin{scope}
\path[clip] ( 49.20, 61.20) rectangle (263.88,239.88);
\definecolor{drawColor}{RGB}{0,0,0}

\path[draw=drawColor,line width= 0.4pt,line join=round,line cap=round] ( 88.63,114.81) circle (  2.25);

\path[draw=drawColor,line width= 0.4pt,line join=round,line cap=round] (144.24,158.36) circle (  2.25);

\path[draw=drawColor,line width= 0.4pt,line join=round,line cap=round] (151.76,197.95) circle (  2.25);

\path[draw=drawColor,line width= 0.4pt,line join=round,line cap=round] (140.26,118.14) circle (  2.25);

\path[draw=drawColor,line width= 0.4pt,line join=round,line cap=round] (239.94,117.08) circle (  2.25);

\path[draw=drawColor,line width= 0.4pt,line join=round,line cap=round] (196.83,145.42) circle (  2.25);

\path[draw=drawColor,line width= 0.4pt,line join=round,line cap=round] (157.44,144.35) circle (  2.25);

\path[draw=drawColor,line width= 0.4pt,line join=round,line cap=round] (133.82,132.14) circle (  2.25);

\path[draw=drawColor,line width= 0.4pt,line join=round,line cap=round] (206.07,146.33) circle (  2.25);

\path[draw=drawColor,line width= 0.4pt,line join=round,line cap=round] (149.40,187.94) circle (  2.25);

\path[draw=drawColor,line width= 0.4pt,line join=round,line cap=round] (204.80,166.96) circle (  2.25);

\path[draw=drawColor,line width= 0.4pt,line join=round,line cap=round] (125.07,152.98) circle (  2.25);

\path[draw=drawColor,line width= 0.4pt,line join=round,line cap=round] (149.27,131.74) circle (  2.25);

\path[draw=drawColor,line width= 0.4pt,line join=round,line cap=round] ( 70.39,213.85) circle (  2.25);

\path[draw=drawColor,line width= 0.4pt,line join=round,line cap=round] (108.62,208.95) circle (  2.25);

\path[draw=drawColor,line width= 0.4pt,line join=round,line cap=round] (111.08,205.55) circle (  2.25);

\path[draw=drawColor,line width= 0.4pt,line join=round,line cap=round] (124.20, 67.82) circle (  2.25);

\path[draw=drawColor,line width= 0.4pt,line join=round,line cap=round] (129.77,120.10) circle (  2.25);

\path[draw=drawColor,line width= 0.4pt,line join=round,line cap=round] (178.33,110.24) circle (  2.25);

\path[draw=drawColor,line width= 0.4pt,line join=round,line cap=round] (116.35,197.77) circle (  2.25);

\path[draw=drawColor,line width= 0.4pt,line join=round,line cap=round] (148.12, 85.75) circle (  2.25);

\path[draw=drawColor,line width= 0.4pt,line join=round,line cap=round] (193.83,149.23) circle (  2.25);

\path[draw=drawColor,line width= 0.4pt,line join=round,line cap=round] (112.90,129.41) circle (  2.25);

\path[draw=drawColor,line width= 0.4pt,line join=round,line cap=round] (123.12,103.61) circle (  2.25);

\path[draw=drawColor,line width= 0.4pt,line join=round,line cap=round] (164.47,115.49) circle (  2.25);

\path[draw=drawColor,line width= 0.4pt,line join=round,line cap=round] (188.40,206.58) circle (  2.25);

\path[draw=drawColor,line width= 0.4pt,line join=round,line cap=round] (140.87,167.89) circle (  2.25);

\path[draw=drawColor,line width= 0.4pt,line join=round,line cap=round] (108.36,149.61) circle (  2.25);

\path[draw=drawColor,line width= 0.4pt,line join=round,line cap=round] (206.20,173.73) circle (  2.25);

\path[draw=drawColor,line width= 0.4pt,line join=round,line cap=round] (163.00,172.74) circle (  2.25);

\path[draw=drawColor,line width= 0.4pt,line join=round,line cap=round] (170.67,171.70) circle (  2.25);

\path[draw=drawColor,line width= 0.4pt,line join=round,line cap=round] (155.26,152.65) circle (  2.25);

\path[draw=drawColor,line width= 0.4pt,line join=round,line cap=round] (109.69, 93.45) circle (  2.25);

\path[draw=drawColor,line width= 0.4pt,line join=round,line cap=round] (104.29,104.54) circle (  2.25);

\path[draw=drawColor,line width= 0.4pt,line join=round,line cap=round] (116.39,187.94) circle (  2.25);

\path[draw=drawColor,line width= 0.4pt,line join=round,line cap=round] (110.19,186.10) circle (  2.25);

\path[draw=drawColor,line width= 0.4pt,line join=round,line cap=round] (154.40,221.24) circle (  2.25);

\path[draw=drawColor,line width= 0.4pt,line join=round,line cap=round] ( 98.77,155.45) circle (  2.25);

\path[draw=drawColor,line width= 0.4pt,line join=round,line cap=round] (111.27,149.20) circle (  2.25);

\path[draw=drawColor,line width= 0.4pt,line join=round,line cap=round] (134.29,156.99) circle (  2.25);

\path[draw=drawColor,line width= 0.4pt,line join=round,line cap=round] (175.49,188.36) circle (  2.25);

\path[draw=drawColor,line width= 0.4pt,line join=round,line cap=round] (121.13,172.66) circle (  2.25);

\path[draw=drawColor,line width= 0.4pt,line join=round,line cap=round] ( 60.85,111.03) circle (  2.25);

\path[draw=drawColor,line width= 0.4pt,line join=round,line cap=round] (170.19,213.60) circle (  2.25);

\path[draw=drawColor,line width= 0.4pt,line join=round,line cap=round] (134.35,111.16) circle (  2.25);

\path[draw=drawColor,line width= 0.4pt,line join=round,line cap=round] (160.20,123.59) circle (  2.25);

\path[draw=drawColor,line width= 0.4pt,line join=round,line cap=round] (126.91,117.48) circle (  2.25);

\path[draw=drawColor,line width= 0.4pt,line join=round,line cap=round] (102.88,216.97) circle (  2.25);

\path[draw=drawColor,line width= 0.4pt,line join=round,line cap=round] (106.88,233.26) circle (  2.25);

\path[draw=drawColor,line width= 0.4pt,line join=round,line cap=round] (152.76,172.04) circle (  2.25);

\path[draw=drawColor,line width= 0.4pt,line join=round,line cap=round] (107.26,116.62) circle (  2.25);

\path[draw=drawColor,line width= 0.4pt,line join=round,line cap=round] (112.11,183.30) circle (  2.25);

\path[draw=drawColor,line width= 0.4pt,line join=round,line cap=round] (164.38,182.43) circle (  2.25);

\path[draw=drawColor,line width= 0.4pt,line join=round,line cap=round] (168.72,152.16) circle (  2.25);

\path[draw=drawColor,line width= 0.4pt,line join=round,line cap=round] (184.50,190.55) circle (  2.25);

\path[draw=drawColor,line width= 0.4pt,line join=round,line cap=round] (182.87,145.33) circle (  2.25);

\path[draw=drawColor,line width= 0.4pt,line join=round,line cap=round] ( 65.77,200.92) circle (  2.25);

\path[draw=drawColor,line width= 0.4pt,line join=round,line cap=round] (211.02,172.09) circle (  2.25);

\path[draw=drawColor,line width= 0.4pt,line join=round,line cap=round] ( 57.15,152.37) circle (  2.25);

\path[draw=drawColor,line width= 0.4pt,line join=round,line cap=round] (136.66,209.72) circle (  2.25);

\path[draw=drawColor,line width= 0.4pt,line join=round,line cap=round] ( 79.14,150.29) circle (  2.25);

\path[draw=drawColor,line width= 0.4pt,line join=round,line cap=round] (174.22,206.27) circle (  2.25);

\path[draw=drawColor,line width= 0.4pt,line join=round,line cap=round] (136.99,159.88) circle (  2.25);

\path[draw=drawColor,line width= 0.4pt,line join=round,line cap=round] (118.28,205.53) circle (  2.25);

\path[draw=drawColor,line width= 0.4pt,line join=round,line cap=round] (159.69, 92.97) circle (  2.25);

\path[draw=drawColor,line width= 0.4pt,line join=round,line cap=round] (132.90,189.44) circle (  2.25);

\path[draw=drawColor,line width= 0.4pt,line join=round,line cap=round] (162.55,176.39) circle (  2.25);

\path[draw=drawColor,line width= 0.4pt,line join=round,line cap=round] (124.19,188.40) circle (  2.25);

\path[draw=drawColor,line width= 0.4pt,line join=round,line cap=round] (132.57,163.59) circle (  2.25);

\path[draw=drawColor,line width= 0.4pt,line join=round,line cap=round] (157.01,130.33) circle (  2.25);

\path[draw=drawColor,line width= 0.4pt,line join=round,line cap=round] (199.59,133.45) circle (  2.25);

\path[draw=drawColor,line width= 0.4pt,line join=round,line cap=round] (255.93,144.85) circle (  2.25);

\path[draw=drawColor,line width= 0.4pt,line join=round,line cap=round] (123.25,205.10) circle (  2.25);

\path[draw=drawColor,line width= 0.4pt,line join=round,line cap=round] (173.20,164.38) circle (  2.25);

\path[draw=drawColor,line width= 0.4pt,line join=round,line cap=round] ( 77.67,132.32) circle (  2.25);

\path[draw=drawColor,line width= 0.4pt,line join=round,line cap=round] ( 98.02, 77.85) circle (  2.25);

\path[draw=drawColor,line width= 0.4pt,line join=round,line cap=round] (138.44,196.58) circle (  2.25);

\path[draw=drawColor,line width= 0.4pt,line join=round,line cap=round] (153.26,164.91) circle (  2.25);

\path[draw=drawColor,line width= 0.4pt,line join=round,line cap=round] (107.42,132.40) circle (  2.25);

\path[draw=drawColor,line width= 0.4pt,line join=round,line cap=round] (188.82,167.11) circle (  2.25);

\path[draw=drawColor,line width= 0.4pt,line join=round,line cap=round] (127.22, 95.76) circle (  2.25);

\path[draw=drawColor,line width= 0.4pt,line join=round,line cap=round] (134.74,200.50) circle (  2.25);

\path[draw=drawColor,line width= 0.4pt,line join=round,line cap=round] (178.35,223.50) circle (  2.25);

\path[draw=drawColor,line width= 0.4pt,line join=round,line cap=round] (191.03,161.60) circle (  2.25);

\path[draw=drawColor,line width= 0.4pt,line join=round,line cap=round] (185.83,156.43) circle (  2.25);

\path[draw=drawColor,line width= 0.4pt,line join=round,line cap=round] (229.74, 86.53) circle (  2.25);

\path[draw=drawColor,line width= 0.4pt,line join=round,line cap=round] (102.69,148.01) circle (  2.25);

\path[draw=drawColor,line width= 0.4pt,line join=round,line cap=round] (170.57,143.31) circle (  2.25);

\path[draw=drawColor,line width= 0.4pt,line join=round,line cap=round] (208.12,127.49) circle (  2.25);

\path[draw=drawColor,line width= 0.4pt,line join=round,line cap=round] (182.93,188.56) circle (  2.25);

\path[draw=drawColor,line width= 0.4pt,line join=round,line cap=round] (112.94,188.30) circle (  2.25);

\path[draw=drawColor,line width= 0.4pt,line join=round,line cap=round] (209.18,198.62) circle (  2.25);

\path[draw=drawColor,line width= 0.4pt,line join=round,line cap=round] (167.81,121.52) circle (  2.25);

\path[draw=drawColor,line width= 0.4pt,line join=round,line cap=round] (204.86,185.61) circle (  2.25);

\path[draw=drawColor,line width= 0.4pt,line join=round,line cap=round] (127.47,180.66) circle (  2.25);

\path[draw=drawColor,line width= 0.4pt,line join=round,line cap=round] (189.94,158.12) circle (  2.25);

\path[draw=drawColor,line width= 0.4pt,line join=round,line cap=round] (116.59,174.45) circle (  2.25);

\path[draw=drawColor,line width= 0.4pt,line join=round,line cap=round] (144.88,201.85) circle (  2.25);

\path[draw=drawColor,line width= 0.4pt,line join=round,line cap=round] (182.74,225.78) circle (  2.25);

\path[draw=drawColor,line width= 0.4pt,line join=round,line cap=round] (136.34,131.66) circle (  2.25);
\end{scope}
\begin{scope}
\path[clip] (  0.00,  0.00) rectangle (289.08,289.08);
\definecolor{drawColor}{RGB}{0,0,0}

\path[draw=drawColor,line width= 0.4pt,line join=round,line cap=round] ( 63.18, 61.20) -- (229.61, 61.20);

\path[draw=drawColor,line width= 0.4pt,line join=round,line cap=round] ( 63.18, 61.20) -- ( 63.18, 55.20);

\path[draw=drawColor,line width= 0.4pt,line join=round,line cap=round] (104.79, 61.20) -- (104.79, 55.20);

\path[draw=drawColor,line width= 0.4pt,line join=round,line cap=round] (146.40, 61.20) -- (146.40, 55.20);

\path[draw=drawColor,line width= 0.4pt,line join=round,line cap=round] (188.00, 61.20) -- (188.00, 55.20);

\path[draw=drawColor,line width= 0.4pt,line join=round,line cap=round] (229.61, 61.20) -- (229.61, 55.20);
Creating temporary TikZ metrics dictionary at:
	/tmp/RtmpSO1HJQ/tikzMetricsDictionary
Using TikZ metrics dictionary at:
	/tmp/RtmpSO1HJQ/tikzMetricsDictionary
Measuring dimensions of: m
Running command: '/usr/bin/pdflatex' -interaction=batchmode -halt-on-error -output-directory '/tmp/RtmpSO1HJQ/tikzDevice51549601c8dbb' 'tikzStringWidthCalc.tex'
Measuring dimensions of: -2
Running command: '/usr/bin/pdflatex' -interaction=batchmode -halt-on-error -output-directory '/tmp/RtmpSO1HJQ/tikzDevice515493be4979a' 'tikzStringWidthCalc.tex'
Measuring dimensions of: \char77
Running command: '/usr/bin/pdflatex' -interaction=batchmode -halt-on-error -output-directory '/tmp/RtmpSO1HJQ/tikzDevice5154940e4dab9' 'tikzStringWidthCalc.tex'

\node[text=drawColor,anchor=base,inner sep=0pt, outer sep=0pt, scale=  1.00] at ( 63.18, 39.60) {-2};
Measuring dimensions of: -1
Running command: '/usr/bin/pdflatex' -interaction=batchmode -halt-on-error -output-directory '/tmp/RtmpSO1HJQ/tikzDevice515491dac083b' 'tikzStringWidthCalc.tex'

\node[text=drawColor,anchor=base,inner sep=0pt, outer sep=0pt, scale=  1.00] at (104.79, 39.60) {-1};
Measuring dimensions of: 0
Running command: '/usr/bin/pdflatex' -interaction=batchmode -halt-on-error -output-directory '/tmp/RtmpSO1HJQ/tikzDevice515493e8f23d6' 'tikzStringWidthCalc.tex'

\node[text=drawColor,anchor=base,inner sep=0pt, outer sep=0pt, scale=  1.00] at (146.40, 39.60) {0};
Measuring dimensions of: 1
Running command: '/usr/bin/pdflatex' -interaction=batchmode -halt-on-error -output-directory '/tmp/RtmpSO1HJQ/tikzDevice51549256f394f' 'tikzStringWidthCalc.tex'

\node[text=drawColor,anchor=base,inner sep=0pt, outer sep=0pt, scale=  1.00] at (188.00, 39.60) {1};
Measuring dimensions of: 2
Running command: '/usr/bin/pdflatex' -interaction=batchmode -halt-on-error -output-directory '/tmp/RtmpSO1HJQ/tikzDevice515494f8129c8' 'tikzStringWidthCalc.tex'

\node[text=drawColor,anchor=base,inner sep=0pt, outer sep=0pt, scale=  1.00] at (229.61, 39.60) {2};

\path[draw=drawColor,line width= 0.4pt,line join=round,line cap=round] ( 49.20, 62.54) -- ( 49.20,221.17);

\path[draw=drawColor,line width= 0.4pt,line join=round,line cap=round] ( 49.20, 62.54) -- ( 43.20, 62.54);

\path[draw=drawColor,line width= 0.4pt,line join=round,line cap=round] ( 49.20, 94.27) -- ( 43.20, 94.27);

\path[draw=drawColor,line width= 0.4pt,line join=round,line cap=round] ( 49.20,125.99) -- ( 43.20,125.99);

\path[draw=drawColor,line width= 0.4pt,line join=round,line cap=round] ( 49.20,157.72) -- ( 43.20,157.72);

\path[draw=drawColor,line width= 0.4pt,line join=round,line cap=round] ( 49.20,189.44) -- ( 43.20,189.44);

\path[draw=drawColor,line width= 0.4pt,line join=round,line cap=round] ( 49.20,221.17) -- ( 43.20,221.17);
Measuring dimensions of: -3
Running command: '/usr/bin/pdflatex' -interaction=batchmode -halt-on-error -output-directory '/tmp/RtmpSO1HJQ/tikzDevice5154923817a45' 'tikzStringWidthCalc.tex'

\node[text=drawColor,rotate= 90.00,anchor=base,inner sep=0pt, outer sep=0pt, scale=  1.00] at ( 34.80, 62.54) {-3};

\node[text=drawColor,rotate= 90.00,anchor=base,inner sep=0pt, outer sep=0pt, scale=  1.00] at ( 34.80, 94.27) {-2};

\node[text=drawColor,rotate= 90.00,anchor=base,inner sep=0pt, outer sep=0pt, scale=  1.00] at ( 34.80,125.99) {-1};

\node[text=drawColor,rotate= 90.00,anchor=base,inner sep=0pt, outer sep=0pt, scale=  1.00] at ( 34.80,157.72) {0};

\node[text=drawColor,rotate= 90.00,anchor=base,inner sep=0pt, outer sep=0pt, scale=  1.00] at ( 34.80,189.44) {1};

\node[text=drawColor,rotate= 90.00,anchor=base,inner sep=0pt, outer sep=0pt, scale=  1.00] at ( 34.80,221.17) {2};

\path[draw=drawColor,line width= 0.4pt,line join=round,line cap=round] ( 49.20, 61.20) --
	(263.88, 61.20) --
	(263.88,239.88) --
	( 49.20,239.88) --
	cycle;
\end{scope}
Measuring dimensions of: rnorm(100)
Running command: '/usr/bin/pdflatex' -interaction=batchmode -halt-on-error -output-directory '/tmp/RtmpSO1HJQ/tikzDevice51549b12457c' 'tikzStringWidthCalc.tex'
\begin{scope}
\path[clip] (  0.00,  0.00) rectangle (289.08,289.08);
\definecolor{drawColor}{RGB}{0,0,0}

\node[text=drawColor,anchor=base,inner sep=0pt, outer sep=0pt, scale=  1.00] at (156.54, 15.60) {rnorm(100)};

\node[text=drawColor,rotate= 90.00,anchor=base,inner sep=0pt, outer sep=0pt, scale=  1.00] at ( 10.80,150.54) {rnorm(100)};
\end{scope}
\end{tikzpicture}

\subsection{org code of the above}
\label{sec:orgbb58e66}

\begin{verbatim}
...
...
...
Once the =tikzDevice= library is installed, you can proceed to evaluate
the following code blocks --- using =C-c C-c= --- :

- Actually load the library
#+begin_src R :session :exports code :results silent
  library("tikzDevice")
#+end_src

- And the magic part

#+name: test_plot
#+begin_src R :session :exports both :results output latex :file test.png
  tikz(console=TRUE, width=4, height=4)
  plot(rnorm(100), rnorm(100))
  dummy <- dev.off()
#+end_src

#+RESULTS: test_plot
#+begin_export latex
% Created by tikzDevice version 0.12.3.1 on 2021-09-10 04:53:27
% !TEX encoding = UTF-8 Unicode
\relax
\begin{tikzpicture}[x=1pt,y=1pt]
\definecolor{fillColor}{RGB}{255,255,255}
\path[use as bounding box,fill=fillColor,fill opacity=0.00] (0,0) rectangle (289.08,289.08);
...
...
...
\end{verbatim}

\subsection{exporting to pdf}
\label{sec:orga5a7dfc}

After all that coding all you have to do is export the \texttt{file.org} to a
\texttt{file.tex} which is then converted to a \texttt{file.pdf} via \texttt{pdflatex} or
another \texttt{latex/tex} engine.

You accomplish that pipeline via the \texttt{org-export-dispatch} ---
you can \texttt{M-x org-export-dispatch} --- and select the \texttt{[l] Export to LaTeX}
option and then \texttt{[o] As PDF file and open}.

\noindent\rule{\textwidth}{0.5pt}

\begin{quote}
If you want some of the functionality of a jupyter notebook within emacs
there you go. Still jupyter has the widgets. But I think this is pretty
cool too.

Happy hacking.

manuel.fuica.morales@gmail.com

10 September 2021
\end{quote}
\end{document}
